\documentclass{beamer}
\usepackage{times}
\usepackage[T1]{fontenc}
\usepackage[parfill]{parskip}
\usepackage[spanish]{babel}
\title{Presentacion del Moogle}
\author{Gabriel Herrera Carrazana}
\date{20/7/23}
\usetheme{Warsaw}
\usecolortheme{crane}
\useoutertheme{shadow}
\useinnertheme{rectangles}
\usepackage{tikz}
\usepackage{pgf-pie}
\usepackage{multicol}

\usepackage{cite}
\bibliographystyle{plain}

\title[Moogle]{Moogle}
\subtitle{Proyecto de Programación}
\author[Gabriel Herrera]{Gabriel Herrera Carrazana}
\institute[MatCom] { Facultad de Matemática y Computación}
\logo{\includegraphics[width=1cm]{matcom.jpg }}



\begin{document}


\itshape






	\frame{\titlepage}
	\begin{frame}{Índice}
		\tableofcontents
	\end{frame}
	
	
	\begin{frame}
	
		
		\frametitle{Moogle}
		\section{¿Qué es el Moogle?}
		
			Moogle es una aplicación web escrita fundamentalmente con el lenguaje de programación C\# . Este nos permite indexar una base de datos de archivos .txt según la relevancia de palabras o frases .  \
		
	\end{frame}
	
	\begin{frame}
		\frametitle{Stack Tecnológico}{Lenguajes mas usados :}
			\section{Stack Tecnológico}
				
		
			\begin{figure}[htb]
			\centering
			\begin{tikzpicture}[scale=0.7]
			\pie{58.8/C\#, 30.2/CSS, 11/HTML}
			\end{tikzpicture}
			\caption{Ejemplo de gráfico de pastel}
			\label{fig:grafico_pastel}
			\end{figure}
		
			
		
	

	\end{frame}
	\begin{frame}
		\frametitle{Clases}
			\section{Clases}
			\begin{multicols}{2}
			
			\subsection{Clases Creadas}
			Clases creadas :
			\begin{itemize}
				\item LoadDocuments.cs
				\item Vector.cs
				\item Matrix.cs
			
			\end{itemize}
			\subsection{Clases Modificadas}
			Clases Modificadas:
				\begin{itemize}
				\item SearchItem.cs
				\item Moogle.cs
				\end{itemize}
					
				
					
			\end{multicols}
				
		
		
	\end{frame}
	\begin{frame}
			\frametitle{Concluiones}
		    \section{Conclusiones}
			
			Finalmente pudimos ver resultados satisfactorios. La aplicación es capáz de cargar hasta 50mb de datos y procesarlos en menos de un minuto. Las búsquedas son completadas en menos de 10 segundos . A pesar de ser un modelo básico de recuperación de datos , Moogle! ha demostrado ser un proyecto tanto instructivo como funcional. 
	\end{frame}
	

\end{document}
	  